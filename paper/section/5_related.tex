\section{Related works}

Many research efforts~\cite{bigcode, big-lessons,big-translation,code-completion,big-predicting} 
have been made to explore how to leverage ``big code'' repositories, 
such as GitHub, BitBucket, and CodePlex, and these works inspire us to explore how to leverage data on VirusTotal. 
SLANG~\cite{code-completion} can fill uncompleted programs with call innovations 
by using statistical models trained from extracted sequences of API calls from large code bases.  
JSNICE~\cite{big-predicting} can predict identifier types and obfuscated identifier names for Javascript programs. 
JSNICE translates programs into dependence graphs and learns a CRF model by using a large training set. 
All predictions are made by optimizing a score function based on the learned CRF model. 
\citet{big-translation} apply phrase-based statistical translation approaches to translate C\# programs to Java.
To sum up, all these techniques are built based on source code repositories, 
and their goals are to improve the development stage. 
However, VirusTotal is a repository containing binary malwares
and the goal for conducting data mining on VirusTotal data is to improve antivirus techniques. 

There are existing works~\cite{hacker-vt,neeles} regarding the conducting of data mining on VirusTotal data to identify malware development cases, 
where malware writers use VirusTotal as a testing platform and 
try to develop malwares that cannot be detected by antivirus engines. 
These techniques utilize submission\_id information, which is different from the information we use. 
We believe other information on VirusTotal could also be leveraged in the future. 


\section{Conclusion}
VirusTotal provides a fruitful opportunity to understand real-world malwares in a large scale.
Unfortunately, it has been largely overlooked by the research community. 
In this paper, we conduct an empirical study on PE malwares on VirusTotal. 
Our study is mainly performed to understand the temporal characteristics and
the family distribution characteristics of malwares. We also build two techniques, 
cache-based malware prediction and hot malware family mining, to validate our observations. 
We expect our work to deepen our understanding of and bring more attention to the data on VirusTotal. 
