\section{Background}

\subsection{VirusTotal}


As a free online service, VirusTotal~\cite{virustotal} analyzes files submitted by real-world users to identify many different kinds of malwares, 
like viruses, worms, trojans, and so on. 
VirusTotal applies different antivirus engines to each submitted file and generate an aggregated reports. 
All submitted files and generated reports are saved and can be accessed through public API. 

The repository on VirusTotal provides a good source to conduct data mining. 
Firstly, there are huge amount of data on VirusTotal.
Figure~\ref{fig:subnum} shows that there were more than 40 million suspicious files 
submitted last November. 
This amount of data makes VirusTotal a rough estimation of malwares in the real world. 
Secondly, all data on VirusTotal are labeled by state-of-the-art antivirus techniques. 
VirusTotal updates each antivirus engine every 5 minutes. 
Besides whether a given a submitted file is detected by an antivirus engine, VirusTotal also keeps exact detection tag returned by each engine. 
There are also online active malware researchers, 
who can comment and vote each submitted file and serve as an important supplement of antivirus engines. 
We believe mining data on VirusTotal could enable many ``Big Security'' applications. 

In industry, antivirus vendors widely use VirusTotal to identify false negatives and false positives of their products. 
They only utilize VirusTotal reports separately for each single suspicious file, but fail to leverage the whole repository. 
In academia, researchers began to pay attention to correlations among different VirusTotal reports. 
For example, {\bf [ToDo: discuss Heqing's work]}
We believe there are much more research opportunities through mining VirusTotal. 


\subsection{Dataset}
\label{sec:meth}


We study all PE malwares submitted in November of 2015. 
In this section, we will firstly discuss how we collect data, 
and then we will present the general characteristics we observe.

\subsection{Data collection}

\begin{table}[h!]
\centering
\footnotesize
{
%\begin{tabular}{@{\hspace{3pt}}l@{\hspace{3pt}}|@{\hspace{3pt}}c@{\hspace{3pt}}}
\begin{tabular}{l|l}
\hline
Metadata Fields & Explanation \\
\hline                            
%\cline{1-1}
{\bf name}      & file name of the submitted sample \\
{\bf timestamp} & timestamp when the submission is conducted \\
{\bf source\_country} & the country where the submission is conducted \\
{\bf source\_id} & user id who conducts the submission\\
{\bf tags} & VirusTotal tag \\
{\bf link} & where to download the submitted sample \\
{\bf size} & file size \\
{\bf type} & file type \\
{\bf first\_seen} & when the same sample was first submitted \\
{\bf last\_seen} & when the same sample was last submitted \\
{\bf hashes} & including sha1, sha256, vhash, md5, and ssdeep\\
{\bf total} & how many engines analyze the sample\\
{\bf positives} & how many engines identify the sample as malicious \\
{\bf positives\_delta} & changes about {\bf positives} fields \\
{\bf report} & detailed detection report from each AV engine \\
%\multicolumn{2}{|l|}
\hline

\end{tabular}
}
\caption{Metadata fields of each submission got from VirusTotal private API.}
\label{tab:fields}
\end{table}

We download submission reports' metadata through private API of VirusTotal.
Table~\ref{tab:fields} shows all metadata fields.
For one report, if its tag field contains either ``peexe'' or ``pedll'', 
we consider the report is about a PE file. 
It is possible that VirusTotal private API returns redundant reports, 
and we use the combination of md5 and timestamp to detect and merge redundant reports.
We only rely on Microsoft antivirus engine to judge whether a submission is malicious or not, 
and which malware family the submitted malware belongs to. 
In total, we collect 43308091 reports and 4732502 PE malwares submitted
in November of 2015. 
The numbers of reports and malwares submitted each day are shown in Figure~\ref{fig:subnum}.

\textit{\underline{Threats to Validity.}}
Similar to all previous empirical study works, all our findings, experimental results, 
and conclusions need to be considered with our methodology in mind. 

VirusTotal private API only tracks which submission reports are sent to each downloader approximately, 
and there is no guarantee that all submission reports on VirusTotal can be downloaded successfully. 
It could be possible that we miss some malwares submitted to VirusTotal. 
We simply leverage Microsoft antivirus engine to decide whether one submission is malicious or not, 
and it is possible that Microsoft antivirus engine cannot make this decision precisely. 
However, how to get a precise label for a PE file is out of scope of this paper.  
Although there are huge mount of malwares on VirusTotal, we do believe that there are malwares never submitted to VirusTotal, 
and there are malwares submitted much later than when they appear in the real world.
However, there are no conceivable ways to study these malwares. 
We believe that malwares in our study provide a representative malware sample of the real world. 

