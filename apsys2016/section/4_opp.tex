\section{Research Opportunities}
\label{sec:oppo}
Beyond the research described above, we illustrate a number of research opportunities as follows: 

{\bf Utilizing other metadata fields.} 
We currently only leverage timestamp and Microsoft detection reports. 
There are many other metadata fields. 
Conducting data mining on these fields could enable 
Many other ``big security'' applications. 
For example, there are existing techniques leverage submission\_id information to identify malware writers~\cite{neeles}. 
For example, future research could leverage ssdeep information to cluster malwares, 
and conduct malware prediction and hot malware mining in a finer granularity. 

{\bf Utilizing other antivirus vendors’ reports.}
We currently only leverage reports from Microsoft, but these reports may not be always accurate.
There are more than 40 antivirus vendors' reports also provided on VirusTotal.
Some vendors’ reports may be better than others, or better than others under some special conditions. 
We leave systematically evaluating all those reports and better combining reports from different vendors in the future. 

{\bf Improving antivirus products.} 
There are also opportunities to improve existing antivirus products. 
For example, many endpoint antivirus softwares, like ClamAV, are built based on a database of malwares' signatures. 
When these antivirus softwares scan suspicious files, all signatures in the database will be checked. 
If we can precisely predict which malware would appear in the near future, 
we could reduce the size of signatures sent to clients' side and also reduce time to check the signature database. 

{\bf Studying other types of malicious files. }
Besides PE files, there are other types of malicious files on VirusTotal, like malicious apps, 
malicious URL, and malicious binary files on other systems. 
What are the characteristics of these files and whether they follow the same patterns as PE files remain an open issue.  